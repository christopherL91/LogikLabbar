\documentclass[a4paper,11pt]{article}
\usepackage[T1]{fontenc}
\usepackage[toc,page]{appendix}
\usepackage[utf8]{inputenc}
\usepackage{lmodern}
\usepackage{listings}
\usepackage{verbatim}
\usepackage{multicol}
\usepackage{csquotes}
\usepackage{fancyhdr}
\usepackage{longtable}
\usepackage{color}
\usepackage{amsmath}
\usepackage{natded}
\usepackage{graphicx}
\usepackage{fancyvrb}
\usepackage{amssymb}
\usepackage{pbox}
\usepackage{wrapfig}
% include code listing options
\include{code}

\newcounter{question}
\setcounter{question}{0}

\title{Labb 2 \\ Logic for computer science}
\author{
  {\bf Christopher Lillthors}\\
  \textbf{911005-3817} \\\\
  {\bf Viktor Kronvall}\\
  \textbf{920225-5478}\\
  \\
  Kurskod: DD1350\\
  %Ev gruppnummer\\
  KTH -- HT14\\
  lillt@kth.se\\
  vkr@kth.se
}

\pagestyle{fancy}
\setlength{\headheight}{54pt}
\fancyfoot[C,R]{\thepage}
\fancyfoot[C]{}
\rhead{\textbf{Labb 2 -- Lillthors \& Kronvall} \\ \date{\today} \\ \ \\}
\lhead{\textbf{Royal Institute of Technology} \\ School of Computer science and communication \\ Civilingenjörsprogrammet Datateknik}
\setlength{\parindent}{0in}
\setlength{\parskip}{0.1in}
\date{\today}
\begin{document}

%Include the Q&A commands
\include{qa}

\maketitle
\thispagestyle{empty}
\begin{abstract}

This paper describes a simple model verification system written in the programming language Prolog.
The system can verify models for a variety of rules given that the input is written using CTL.
\end{abstract}
\renewcommand{\arraystretch}{1.2}
\newpage
\thispagestyle{empty}
\tableofcontents
\newpage
\clearpage
\setcounter{page}{1}

\section{Example models}
\label{models}
We construct a model $\mathcal{M}$ for a web client that is accessing a server which requires authentication and authorization
before some of the resources hosted by the server may be accessed.
\begin{figure}[ht]
\caption{Graph of the model}
\includegraphics[scale=0.6]{model.pdf}
\label{fig:graph}
\end{figure}

\newpage
\subsection{Valid assumption}
Given the state \textit{logged in} assume that $EF(\neg logged \: in)$ holds.

This formula describes the condition that given a logged in state, there is a path
to reach a state where one is not logged in.

By analyzing the graph in figure \ref{fig:graph} we can see that one possible state that is reachable from
\textit{logged in} is the state \textit{start} where one is not logged in. Therefore we can conclude that this is intuitively correct.

\subsection{Invalid assumption}
Given the state \textit{login response} assume that $AX(logged \: in)$ holds.

This formula describes the condition that given a login response state, the next state must contain the propositional variable \textit{logged in}, i.e $L(s)=logged \: in$.

By analyzing the graph in figure \ref{fig:graph} and looking at the paths from \textit{login response} there is one path that describes a failed login where one is sent back to the \textit{start} state. Given this we deduce that the assumption does not hold as $logged \: in \notin L(start)$.

\section{Implementation}

The following rules were implemented for models.
\newline
\includegraphics[scale=0.48]{rules.png}
\newpage
\subsection{Predicates for rules}
In the following table of predicates, a list of propositional letters $L$, a list of visited states along the path $U$ and a predicate $Neighbor(a,b)$ that solves if $a$ and $b$ are neighbors, are assumed.
\begin{center}
\begin{tabular}{ | l | c | }
  Rule & Condition \\
  \hline
  $Check(s,p)$ & $p \in L(s)$ \\
  $Check(s,\neg p)$ & $p \notin L(s)$ \\
  $And(s,\phi,\psi) $ & $Check(s,\phi) \land Check(s,\psi)$ \\
  $Or1(s,\phi,\psi) $ & $Check(s,\phi)$ \\
  $Or2(s,\phi,\psi) $ & $Check(s,\psi)$ \\
  $Or(s,\phi,\psi) $ & $Or1(s,\phi,\psi) \lor Or2(s,\phi,\psi)$ \\
  $AX(s,\phi) $ & $\forall x : Neighbor(s,x)\land Check(x,\phi)$ \\
  $AG1(U,s,\phi) $ & $s \in U$ \\[10pt]
  $AG2(U,s,\phi) $ & \pbox[t]{11cm}{\centering$s \notin U \land Check(s,\phi) \land (\forall x : Neighbor(s,x)\land AG([U|s],x,\phi) \land Check(x,\phi))$} \\[20pt]
  $AG(U,s,\phi) $ & $AG1(U,s,\phi) \lor AG2(U,s,\phi)$ \\
  $AF1(U,s,\phi) $ & $s \notin U \land Check(s,\phi)$ \\
  $AF2(U,s,\phi) $ & $s \notin U \land (\forall x : Neighbor(s,x)\land AF(U,x,\phi))$ \\
  $AF(U,s,\phi) $ & $AF1(U,s,\phi) \lor AF2(U,s,\phi)$ \\
  $EX(U,s,\phi)$ & $\exists x(Neighbor(x,s) \land Check(x,\phi))$ \\
  $EG1(U,s,\phi)$ & $s \in U$ \\
  $EG2(U,s,\phi)$ & $s \notin U \land Check(s,\phi) \land \exists x (Neighbor(x,s) \land EG(U,s,\phi) )$ \\
  $EG(U,s,\phi)$ & $EG1(U,s,\phi) \lor EG2(U,s,\phi)$ \\
  $EF1(U,s,\phi)$ & $s \notin U \land Check(s,\phi)$ \\
  $EF2(U,s,\phi)$ & $s \notin U \land \exists x (Neighbor(x,s) \land EF([U|s],x,\phi))$ \\
  $EF(U,s,\phi)$ & $EF1(U,s,\phi) \lor EF2(U,s,\phi)$ \\
\end{tabular}
\end{center}

The model checking with the above predicates was then implemented in the Prolog programming language by pattern matching the several rules. Also, in the prolog implementation all the rules were recursively matched as an argument to the $Check$ predicate in order to allow for a rule that is called on another rule.

\section{Results}

Both of the example models in section \ref{models} were correctly handled by the Prolog implementation yielding a $true$ result for the valid assumption and a $false$ result for the invalid one.

In addition to that the 733 test cases given by KTH also passed with correct results.

\section{Questions}

\Que{What's the difference between your implementation and the implementation written in the text book?}
%TODO
\Ans{
  % contradictions
  % until
  % implications
  The implementation found in the text book supports more rules than our does.
  Contradiction ($\bot$), until ($U$) , implication ($\to$) and tautology ($\top$) is not supported by our implementation.
  % Contradiction would be used in a situation like this $\{p,\neg p\} \in L(s)$ and would return $true$.
  % Until is used to describe all the states which does not
}

% Tvetydligt ifall man har flera labels? E[false, U q]
% det finns en väg sådant att varje state är falsk tills dess att s_n(q) är uppfyllt.
% s_m(not(k),not(l))

\Que{How did you handle a variable number of premises in rules?}
\Ans{We use the built in predicate \textit{member} to iterate over all the neighbors.}

\Que{How big models can you handle with your implementation?}
\Ans{The time complexity $\mathcal{O}(n^p)$ increases by a factor for each time a rule that is using all neighbors to a state. That means that $p$ is increased by one for each time and we quickly approach a cubic asymptotic progression of the time $\mathcal{O}(n^3)$. This leads to an untenable situation for solving formulas in a reasonable amount of time for bigger models. For example, a model that has $10^3$ neighbors for each state, progressing only three states from the starting state will result in over $10^9$ checks being performed.}

\newpage
\begin{appendices}
\section{Source code}
\lstinputlisting{labb2.pl}
\section{Valid assumption}
\verbatiminput{valid_model.txt}
\section{Invalid assumption}
\verbatiminput{invalid_model.txt}
\end{appendices}
\end{document}