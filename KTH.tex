\documentclass[a4paper,11pt]{article}
\usepackage[T1]{fontenc}
\usepackage[toc,page]{appendix}
\usepackage[utf8]{inputenc}
\usepackage{lmodern}
\usepackage{listings}
\usepackage{verbatim}
\usepackage{multicol}
\usepackage{csquotes}
\usepackage{fancyhdr}
\usepackage{longtable}
\usepackage{color}
\usepackage{amsmath}
\usepackage{natded}
\usepackage{graphicx}
\usepackage{amssymb}
\usepackage{wrapfig}
% include code listing options
% define colors for code listings
\definecolor{mygreen}{rgb}{0,0.6,0}
\definecolor{mygray}{rgb}{0.5,0.5,0.5}
\definecolor{mymauve}{rgb}{0.58,0,0.82}

% define options for code listings
\lstset{ %
  backgroundcolor=\color{white},   % choose the background color; you must add \usepackage{color} or \usepackage{xcolor}
  basicstyle=\footnotesize,        % the size of the fonts that are used for the code
  breakatwhitespace=false,         % sets if automatic breaks should only happen at whitespace
  breaklines=true,                 % sets automatic line breaking
  captionpos=b,                    % sets the caption-position to bottom
  commentstyle=\color{mygreen},    % comment style
  deletekeywords={...},            % if you want to delete keywords from the given language
  escapeinside={\%*}{*)},          % if you want to add LaTeX within your code
  extendedchars=true,              % lets you use non-ASCII characters; for 8-bits encodings only, does not work with UTF-8
  frame=single,                    % adds a frame around the code
  keepspaces=true,                 % keeps spaces in text, useful for keeping indentation of code (possibly needs columns=flexible)
  keywordstyle=\color{blue},       % keyword style
  language=Prolog,                 % the language of the code
  morekeywords={*,...},            % if you want to add more keywords to the set
  numbers=left,                    % where to put the line-numbers; possible values are (none, left, right)
  numbersep=5pt,                   % how far the line-numbers are from the code
  numberstyle=\tiny\color{mygray}, % the style that is used for the line-numbers
  rulecolor=\color{black},         % if not set, the frame-color may be changed on line-breaks within not-black text (e.g. comments (green here))
  showspaces=false,                % show spaces everywhere adding particular underscores; it overrides 'showstringspaces'
  showstringspaces=false,          % underline spaces within strings only
  showtabs=false,                  % show tabs within strings adding particular underscores
  stepnumber=1,                    % the step between two line-numbers. If it's 1, each line will be numbered
  stringstyle=\color{mymauve},     % string literal style
  tabsize=2,                       % sets default tabsize to 2 spaces
  title=\lstname                   % show the filename of files included with \lstinputlisting; also try caption instead of title
}

\title{Labb 1 \\ Logic for computer science}
\author{
  {\bf Christopher Lillthors}\\
  \textbf{911005-3817} \\\\
  {\bf Viktor Kronvall}\\
  \textbf{920225-5478}\\
  \\
  Kurskod: DD1350\\
  %Ev gruppnummer\\
  KTH -- HT14\\
  lillt@kth.se\\
  vkr@kth.se
}

\pagestyle{fancy}
\setlength{\headheight}{54pt}
\fancyfoot[C,R]{\thepage}
\fancyfoot[C]{}
\rhead{\textbf{Labb 1 -- Lillthors \& Kronvall} \\ \date{\today} \\ \ \\}
\lhead{\textbf{Royal Institute of Technology} \\ School of Computer science and communication \\ Civilingenjörsprogrammet Datateknik}
\setlength{\parindent}{0in}
\setlength{\parskip}{0.1in}
\date{\today}
\begin{document}
\maketitle
\thispagestyle{empty}
\begin{abstract}
Hello world
\end{abstract}
\renewcommand{\arraystretch}{1.2}
\newpage
\thispagestyle{empty}
\tableofcontents
\newpage
\clearpage
\setcounter{page}{1}
\section{Proofs}

A simple valid proof.
\[
\Jproof{
	\cablk{
		\proofline{p \land (q \lor r)}{premise}
		\proofline{p}{$\land e_1 \: 1$}
		\proofline{q \lor r}{$\land e_2 \: 1$}
		\cablk{
			\proofline{q}{assumption}
			\proofline{p \land q}{$\land i \: 2,4$}
			\proofline{(p \land q)\lor (p \land r)}{$\lor i_{1} \: 5$}
		}
		\cablk{
			\proofline{r}{assumption}
			\proofline{p \land r}{$\land i \: 2,7$}
			\proofline{(p \land q)\lor (p \land r)}{$\lor i_{2} \: 8$}
		}
		\proofline{(p \land q)\lor (p \land r)}{$\lor e \: 3,4-6,7-9$}
	}
}
\]


A simple invalid proof.
\[
\Jproof{
	\cablk{
		\proofline{p \land (q \lor r)}{premise}
		\proofline{p}{$\land e_1 \: 1$}
		\proofline{q \lor r}{$\land e_2 \: 1$}
		\cablk{
			\proofline{q}{assumption}
			\proofline{r}{assumption}
			\proofline{p \land q}{$\land i \: 2,4$}
			\proofline{p \land r}{$\land i \: 2,5$}
			\proofline{(p \land q) \land (p \land r)}{$\land i_1 \: 6,7$}
		}
		\proofline{q \to (p \land q) \land (p \land r)}{$\to i \: 4-8$}
	}
}
\]


\section{Rules}
Table of rules
\begin{center}
    \begin{longtable}{ l p{10cm}}
    \hline
    \textbf{Rule} & \textbf{True When}\\ \hline
    premise & always \\ \hline
    assumption & start of a sub proof \\ \hline
    row $x$ & x is lesser than current row \\ \hline
    copy $x$ & row x exists \\ \hline
    $\land i \; x,y$ & rows x and y exists and contains the left and right operands respectively \\ \hline
    $\land e_1 \; x$ & row x exists and contains a statement containing $\land$ with the result preceeding the operator \\ \hline
	$\land e_2 \; x$ & row x exists and contains a statement containing $\land$ with the result succeeding the operator \\ \hline
	$\lor i_1 \; x$ & row x exists and contains the left operand of the resulting $\lor$ operation. \\ \hline
	$\lor i_2 \; x$ & row x exists and contains the right operand of the resulting $\lor$ operation. \\ \hline
	$\lor e \; x,y-u,v-w$ & all rows exists and $y \leq u$ and $v \leq w$ and rows u and w both containing the result \\ \hline
	$\to i \; x-y$ & $x \leq y$ row x contains an assumption with the left operand and row y the right operand \\ \hline
	$\to e \; x,y$ & row y contains an implication and row x contains that implication's left operand and the result is the right operand \\ \hline
	$\neg i \; x-y$ & $x \leq y$ and row x contains an assumption and row y contains a contradiction \\ \hline
	$\neg e \; x,y$ & row y contains the negation of row x and the result is a contradiction \\ \hline
	$\bot e \; x$ & row x contains a contradiction \\ \hline
	$\neg \neg i \; x$ & the result contains the double negation of row x \\ \hline
	$\neg \neg e \; x$ & row x contains the double negation of the result \\ \hline
	MT $x,y$ & row x contains $a \to b$ and row y $\neg b$ and the result is $\neg a$ \\ \hline
	PBC $x-y$ & $x \leq y$ and row x contains an assumption for $\neg a$ and row y contains a contradiction and the result is a \\ \hline
	LEM & the result is $a \lor \neg a$ \\ \hline
    \end{longtable}
\end{center}
For all ranges $x-y$ in the above table both row x and row y are inside a sub proof (box).

\section{Algorithm}
The first step of the proof checking is to start at either the first or the last row. In this example we will only choose the first row.

For that row we ensure that the given rule is valid by checking that no index refers to a row that is below the current row and if we have a rule that refers to a sub proof, we ensure that it only checks the first and last rows of the range referring to that sub proof.

For every rule we apply the definition and confirm that it holds considering the result on the current row and all referred rows specified by indices. We then proceed to the following row and apply the same rule validation for that row until we reach the end of the proof.

For all the following rows we apply the same rule validation as above.
\section{Implementation}
In the Prolog implementation of the proof validation the proof is reversed in order to be able to utilize the head-tail functionality of the language. While reversed the list is then traversed line by line checking the rule for every line. Because the head is exhausted it is then not necessary to ensure that no index refers to a previous row as those are eliminated when continuing the traversal of the proof.

In case that the next line is a part of sub proof, the sub proof is then expanded and the rest of the proof is handled as if the sub proof was a part of the main proof.

Invalid assumptions that are found in a pre-processing step before checking the validity of the other properties of the proof. An assumption is considered invalid if it is in any other position than the first of a subproof.

The validity of other rules are ensured by pattern-matching both the rule (right hand side) and the result (left hand side) with the rows referred to by the indicies in the rule. For example $\land e_i$ rule validation is performed by matching the left operand of the referred row with the result of the current row.
\newpage
The implementation for this rule in Prolog is:

\begin{lstlisting}[frame=single,language=Prolog]
%signature: valid_rule(Rule,Result,Rows,Premises).
valid_rule(andel1(RX),X,Rows,_):- row(Rows, RX, and(X,_),_).
\end{lstlisting}

Since all rules are given in prefix notation, i.e. the rule name is always preceding all arguments and as such can without transformation be implemented using predicates.
\section{Results}
The Prolog implementation
\section{Discussion}
Note that while all given tests pass but there are some cases where the implementation still might fail.

It's not an easy job to add more rules into the actual implementation, since more and more rules depend on each other.
In order to add more rules there's a big need for rewriting the implementation.

While we implemented the algorithm we thought about the most time consuming
\section{Conclusion}
We have implemented a method for testing if natural deduction is possible for a given proof and the validity of the proof.
\newpage
\begin{appendices}
\section{Source code}
\lstinputlisting{labb1.pl}
\end{appendices}
\end{document}